\newcommand{\D}{\vphantom{\ddots}}
\renewcommand{\thesection}{\Alph{section}}

\chapter*{Annexe}
\addcontentsline{toc}{chapter}{Annexe}
\section{GMRES n'est pas une application linéaire}\label{annexe:gmres_non-linear}

\subsection*{Contexte :}
  \paragraph{}
  On se place dans $\mathbb{K}^n$ avec $\mathbb{K} = \mathbb{R}\textrm{ ou }\mathbb{C}$ et $n \in \mathbb{N^*}$.
  On considère une matrice carrée $A\in\matrixsymb{n}{K}$ inversible, un vecteur $b\in\mathbb{K}^n$, et on cherche à résoudre le système linéaire :
  \[Ax = b\]
  pour obtenir le vecteur solution $x\in\mathbb{K}^n$.
  On utilise pour cela une méthode itérative de type GMRES.
  On utilise par exemple $k$-étapes de GMRES avec un itéré initial nul.

  \paragraph{}
  Pour $n \in \mathbb{N^*}$, $k \in [\![1, n[\![$ et $A\in\operatorname{GL}_n\left(\mathbb{K}\right)$, on défini l'application correspondante
  \begin{align*}
    \operatorname{GMRES}_{A, k}\quad : \quad \mathbb{K}^n &\to \mathbb{K}^n \\
    b \;&\mapsto x_k\textrm{, l'itéré produit à l'itération k.}
  \end{align*}
  Cette application est bien définie, car si la solution est atteinte en $k' < k$ on considère que GMRES stagne en $x_k = x_{k'} = A^{-1}b$.


\subsection*{Non-linéarité :}

  \paragraph{}
  Montrons par un contre-exemple que pour $k < n$, $\operatorname{GMRES}_{A, k}$ n'est pas nécessairement linéaire en sa deuxième variable.
  On note que si $k = n$, GMRES donne la solution exacte, donc peut être représenté par la matrice $A^{-1}$, et donc est linéaire.

  \paragraph{}
  Posons
  \[A = \begin{bmatrix}
    1 &        & \D     &        & 1 \\
    1 & 0      & \D     &        &   \\
      & \ddots & \ddots &        &   \\
    \D& \D     & \ddots & \ddots & \D\\
      &        & \D     & 1      & 0
  \end{bmatrix} \in \matrixsymb{n}{K}\ .\]
  On a alors :
  \begin{align*}
    Ae_1 &= e_1 + e_2 \\
    Ae_i &= e_{i+1} \qquad \left(i = 2, \dots, n-1\right) \\
    Ae_n &= e_1
  \end{align*}
  où les $e_i$ sont les vecteurs de la base canonique de $\mathbb{K}^n$.
  La matrice $A$ est bien inversible, de déterminant $\left(-1\right)^{n+1} \neq 0$.


\subsection*{\underline{$b = e_1$} :}

  \paragraph{}
  On remarque que : pour $0 \le i < k,\; A^ib = \sum_{j=1}^{i+1} e_j$.
  Alors, l'espace de Krylov construit par l'itération d'Arnoldi est : $\krylov[A, b]{k} = \operatorname{Vect}\left(e_1, \dots, e_k\right)$.
  Ainsi, on peut écrire pour $x\in\krylov[A, b]{k} : x = \sum_{i=1}^k \lambda_i e_i$.
  GMRES minimise la norme du résidu :
  \begin{align*}
    \norm{r_k}^2 &= \norm{b - Ax}^2 \\
                 &= \norm{e_1 - \sum_{i=1}^k \lambda_i Ae_i}^2 \\
                 &= \norm{e_1 - \lambda_1 \left(e_1 + e_2\right) - \sum_{i=2}^k \lambda_i e_{i+1}}^2 \\
                 &= \left|1 - \lambda_1\right|^2 + \left|\lambda_1\right|^2 + \sum_{i=2}^k \left|\lambda_i\right|^2
  \end{align*}

  \paragraph{}
  On voit que la norme du résidu est minimisée sur l'espace de Krylov si $\lambda_{i\geq 2} = 0$ et $\lambda_1 = 1/2$.
  Ainsi, $\operatorname{GMRES}_{A, k}\left(e_1\right) = \frac{1}{2}e_1$.

\subsection*{\underline{$b = e_2$} :}

  \paragraph{}
  On remarque que : pour $0 \le i < k,\quad A^ib = e_{i + 2}$.
  Alors, l'espace de Krylov construit par l'itération d'Arnoldi est : $\krylov[A, b]{k} = \operatorname{Vect}\left(e_2, \dots, e_{k+1}\right)$.
  Ainsi, on peut écrire pour $x\in\krylov[A, b]{k} : x = \sum_{i=2}^{k+1} \lambda_i e_i$.
  GMRES minimise la norme du résidu :
  \begin{align*}
    \norm{r_k}^2 &= \norm{b - Ax}^2 \\
                 &= \norm{e_2 - \sum_{i=2}^{k+1} \lambda_i Ae_i}^2 \\
                 &= \norm{e_2 - \sum_{i=2}^{k+1} \lambda_i e_{i+1}}^2 \qquad\textrm{en notant $e_{n+1} = e_1$, si nécessaire (si $k = n-1$)}\\
                 &= 1 + \sum_{i=2}^{k+1} \left|\lambda_i\right|^2
  \end{align*}

  \paragraph{}
  On voit que la norme du résidu est minimisée sur l'espace de Krylov si les $\lambda_{i}$ sont nuls.
  Ainsi, $\operatorname{GMRES}_{A, k}\left(e_2\right) = 0$.

\subsection*{\underline{$b = e_1 + e_2$} :}

  \paragraph{}
  On remarque que : pour $0 \le i < k,\; A^ib = \sum_{j=1}^{i+2} e_j$.
  Alors, l'espace de Krylov construit par l'itération d'Arnoldi est : $\krylov[A, b]{k} = \operatorname{Vect}\left(e_1 + e_2, e_3,\dots, e_{k+1}\right)$.
  Ainsi, on peut écrire pour $x\in\krylov[A, b]{k} : x = \lambda_1\left(e_1 + e_2\right) + \sum_{i=3}^{k+1} \lambda_i e_i$.
  GMRES minimise la norme du résidu :
  \begin{align*}
    \norm{r_k}^2 &= \norm{b - Ax}^2 \\
                 &= \norm{e_1 + e_2 - \lambda_1 A\left(e_1 + e_2\right) - \sum_{i=3}^{k+1} \lambda_i Ae_i}^2
  \end{align*}
  On distingue alors deux cas :
  \begin{itemize}
    \item si $k < n-1$ :
      \begin{align*}
        \norm{r_k}^2 &= \norm{\left(1 - \lambda_1\right)\left(e_1 + e_2\right) - \lambda_1 e_3 - \sum_{i=3}^{k+1} \lambda_ie_{i+1}}^2 \\
                     &= 2\left|1 - \lambda_1\right|^2 + \left|\lambda_1\right|^2 + \sum_{i=3}^{k+1} \left|\lambda_i\right|^2
      \end{align*}

      \paragraph{}
      On voit que la norme du résidu est minimisée sur l'espace de Krylov si $\lambda_{i\ge 3} = 0$ et $\lambda_1 = 2/3$.
      Ainsi, $\operatorname{GMRES}_{A, k}\left(e_1 + e_2\right) = \frac{2}{3}e_1 + \frac{2}{3}e_2$.
    \vspace{\baselineskip}
    \item si $k = n-1$ :
      \begin{align*}
        \norm{r_k}^2 &= \norm{\left(1 - \lambda_1\right)\left(e_1 + e_2\right) - \lambda_1 e_3 - \sum_{i=3}^{n-1} \lambda_ie_{i+1} - \lambda_n e_1}^2 \\
                     &= \left|1 - \lambda_1 - \lambda_n\right|^2 + \left|1 - \lambda_1\right|^2 + \left|\lambda_1\right|^2 + \sum_{i=3}^{n-1} \left|\lambda_i\right|^2
      \end{align*}

      \paragraph{}
      On voit que la norme du résidu est minimisée sur l'espace de Krylov si $\lambda_{3\le i < n} = 0$ et $\lambda_1 = \lambda_n = 1/2$.
      Ainsi, $\operatorname{GMRES}_{A, k}\left(e_1 + e_2\right) = \frac{1}{2}e_1 + \frac{1}{2}e_2 + \frac{1}{2}e_n$.
  \end{itemize}


\paragraph{}
En conclusion, on voit que $\operatorname{GMRES}_{A, k}\left(e_1 + e_2\right) \neq \operatorname{GMRES}_{A, k}\left(e_1\right) + \operatorname{GMRES}_{A, k}\left(e_2\right)$, ce qui prouve que $\operatorname{GMRES}_{A, k}$ n'est pas une application linéaire, et n'admet pas de représentation matricielle.
Pour cette raison, la théorie développée autour du préconditionnement "classique" ne s'applique pas, et on utilisera GMRES comme un préconditionnement flexible.
