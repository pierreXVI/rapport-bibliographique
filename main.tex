\documentclass[a4paper]{article}
\usepackage[utf8]{inputenc}
\usepackage[french]{babel}
\usepackage[T1]{fontenc}
\usepackage{geometry}
\usepackage{graphicx}

\usepackage{amssymb}
\usepackage{amsmath}
%\usepackage{mathtools}
\usepackage{ifthen}
\usepackage{xcolor}

%\usepackage{url}
\usepackage{caption}
\usepackage{subcaption}


\newcommand{\CEDRE}{\emph{CEDRE}}
\newcommand{\PETSc}{\emph{PETSc}}

%\newcommand{\transpose}[2][-3mu]{\ensuremath{\mskip1mu\prescript{\smash{\mathrm t\mkern#1}}{}{\mathstrut#2}}}
\newcommand{\matrixsymb}[2]{\mathcal{M}_{#1}\left(\mathbb{#2}\right)}
%\newcommand{\orthogroup}[2]{\operatorname{O}_{#1}\left(\mathbb{#2}\right)}
%\newcommand{\adj}[1]{{#1}^*}
\newcommand{\krylov}[2][]{\ifthenelse{\equal{#1}{}}{\mathcal{K}_{#2}}{\mathcal{K}_{#2}\left({#1}\right)}}
\newcommand{\norm}[2][]{\ifthenelse{\equal{#1}{}}{\left\|{#2}\right\|}{\left\|{#2}\right\|_{{#1}}}}
% \renewcommand{\parallel}{{\mathbin{/\mkern-5mu/}}}


\newcommand{\GP}[1]{{\color{red}   #1}}
\newcommand{\LM}[1]{{\color{green} #1}}
\newcommand{\PS}[1]{{\color{blue}  #1}}


\title{Rapport bibliographique :\\[\baselineskip]\large Évaluation de librairies d'algèbre linéaire pour l'obtention efficace de solutions stationnaires dans les domaines de l'énergétique et de la multi-physique}
\author{Pierre Seize}
\date{}

\begin{document}
\pagenumbering{Alph}
\maketitle
\pagenumbering{arabic}
\tableofcontents

\chapter{Introduction}

\section{Contexte d'étude}

	\paragraph{}
	La nécessité de la simulation numérique est aujourd'hui bien admise, tant dans le monde industriel que le monde académique.
	Les entreprises comme les laboratoires de recherche ont besoin de pouvoir accéder à certaines grandeurs physiques associées à des phénomènes et à des régimes de fonctionnement bien particuliers.
	Il arrive souvent que ces régimes ne soient pas réalisables à notre échelle, en raison de limitations matérielles ou financières.
	On peut prendre comme exemple l'étude du givrage qui a lieu sur la voilure d'un avion, qui est réalisable expérimentalement mais représente un budget imposant pour l’avionneur, ou bien l'étude des transferts thermiques d'une capsule de rentrée atmosphérique, bien plus difficile à réaliser expérimentalement.
	Pour contourner ces limitations, la simulation numérique est la meilleure option, car elle permet de modéliser un tel cas d'étude par l’exécution d'un programme informatique, et d'obtenir un ensemble important de données qui seront analysées par la suite pour répondre aux questions souhaitées.

	\paragraph{}
	La simulation numérique n'est pas non plus un outil absolu ou parfait : si elle rend possible l'obtention de données inaccessible par l'expérimentation, elle nécessite cependant un travail de modélisation et de développement, des connaissances, et surtout un important travail de calcul informatique.
	Par travail, on peut entendre ici une puissance de calcul multipliée par un temps de calcul : n'importe quelle machine ne peut pas réaliser n'importe quelle simulation, certains cas nécessitent des machines très puissantes, et même avec ces machines la simulation numérique n'est pas instantanée, et il est courant de lancer des calculs durant plusieurs semaines.
	Dans certains cas, comme pour les écoulements turbulents à grands nombres de Reynolds, la simulation de toutes les échelles de la turbulence est même inaccessible.

	\paragraph{}
	Pour éviter de se retrouver avec des calculs qui s'exécuteraient trop longtemps sur des machines trop puissantes, ce qui coûterait trop cher, on va réaliser des choix pour optimiser l'efficacité du calcul.
	On va privilégier un type d'algorithme pour un type de calcul, choisir une mise en donnée plus adaptée qu'une autre pour notre architecture logicielle, ..., tout cela pour réduire le coût en temps et en puissance de calcul de notre simulation numérique.
	C'est pour pouvoir effectuer ces choix qu'il faut posséder des connaissances dans ce domaine.
	L'objectif de ma thèse est d'identifier et de comparer de tels choix qui vont rendre plus efficace la simulation de phénomènes physiques s'inscrivant dans un cadre donné : les problèmes stationnaires en énergétique et multi-physique.

	\paragraph{}
	Il est d'usage de séparer les problèmes de simulation numérique en deux grandes classes : les problèmes instationnaires qui vont décrire l'évolution d'un système au cours du temps, et les problèmes stationnaires qui cherchent la valeur d'un état convergé du même système.
	Cette distinction est déjà bien connue, et engendre déjà différents choix dans la résolution des problèmes : un problème instationnaire aura un faible pas de temps pour bien capter l'évolution temporelle de chacune des physiques mises en jeu, et utilisera des méthodes d'intégration temporelles explicites, alors qu'un problème stationnaire permet une montée en CFL qui s'accompagne de l'utilisation de méthodes implicites.
	Ma thèse s'intéresse au second type : les problèmes stationnaires.


	\subsection{Le problème physique}

		\paragraph{}
		On considère un problème type de simulation numérique en dynamique des fluides (CFD), qui consiste à résoudre sur un domaine spatial donné une ou un ensemble d'équations décrivant les lois d'évolution du modèle physique.
		Typiquement, on cherche à résoudre une équation aux dérivées partielles :
		\begin{equation}\label{eq:edp_0}
			\frac{\partial\textup{w}}{\partial t} = \operatorname{f}\left(\textup{w}\right)
		\end{equation}
		sur un domaine spatial $\mathcal{D}$.
		Concrètement, $\textup{w}$ est un vecteur représentant les grandeurs physiques du système en un point de l'espace et $\operatorname{f}$ une fonction dépendant de l'état du système.

		\paragraph{}
		L'équation (\ref{eq:edp_0}) découle de la physique du problème à résoudre.
		Par exemple, pour l'équation de la chaleur sans terme source, on prendrait pour l'état du système $\textup{w} = T$ la température, et pour le second membre $\operatorname{f}\left(T\right) = D\nabla^2T$ avec $D$ le coefficient de diffusion thermique.
		Un autre exemple, qui est lui dans le thème de ma thèse, est le cas des équations d'Euler sans terme source pour un gaz parfait.
		Si pour simplifier on se place dans un cas à une dimension, on prend l'état du système comme $\textup{w} = \transpose{\left(\rho, \rho u, \rho E\right)} = \transpose{\left(\textup{w}_1, \textup{w}_2, \textup{w}_3\right)}$.
		Dans ce cas, avec la relation de fermeture $\rho E = \frac{p}{\gamma - 1} + \frac{\rho u^2}{2}$, le second membre s'écrit :
		\[\operatorname{f}\left(\begin{aligned}\textup{w}_1\\\textup{w}_2\\\textup{w}_2\end{aligned}\right)
			= -\partial_x\begin{pmatrix}
				\rho u\\
				\rho u^2 + p\\
				\left(E + p\right)\rho u
				\end{pmatrix}
			= -\partial_x\begin{pmatrix}
				\textup{w}_2\\
				\frac{\left(3 - \gamma\right)\textup{w}_2^2}{\textup{w}_1} + \left(\gamma - 1\right)\textup{w}_3\\
				\left(\gamma\textup{w}_3 - \left(\gamma - 1\right)\frac{\textup{w}_2^2}{2\textup{w}_1}\right)\frac{\textup{w}_2}{\textup{w}_1}
			\end{pmatrix}
		\ .\]

		\paragraph{}
		Dans notre cadre d'étude, le second membre intervenant dans l'équation (\ref{eq:edp_0}) met en jeu des dérivées spatiales, comme on peut le voir sur les deux exemples donnés précédemment.
		Le calcul de ces dérivées spatiales dépend de la discrétisation spatiale choisie.


	\subsection{Le problème numérique}

		\paragraph{}
		Par différents procédés mathématiques, l'équation physique (\ref{eq:edp_0}) est transformée en une autre équation que l'on pourrait qualifier d'équation discrétisée.
		C'est cette équation que l'on résout en pratique.

		\paragraph{}
		Je m'intéresse dans le cadre de ma thèse aux problèmes stationnaires.
		Pour un problème stationnaire, l'état du système n'évolue pas, et donc l'équation à résoudre devient :
		\begin{equation}\label{eq:f=0}
			\operatorname{f}\left(\textup{w}\right) = 0\ .
		\end{equation}
		Puisqu'on cherche une solution stationnaire, on l'exprime comme un zéro de cette fonction $\operatorname{f}$.

		\paragraph{}
		Pour obtenir une solution de (\ref{eq:f=0}), on introduit la notion de pseudo-temps \cite{KelleyKeyes1996}.
		En effet, utiliser un algorithme tel que la méthode de Newton de manière brute sur la fonction $\operatorname{f}$ ne permet pas d'obtenir la solution en général.
		Pour les cas qui nous intéressent, le second membre est fortement non-linéaire, et donc une simple méthode de Newton ne convergera pas toujours.
		L'autre possibilité est de résoudre le problème instationnaire en laissant le système converger vers sa solution stationnaire.
		Cependant, résoudre l'évolution du système en fonction du temps depuis un état initial connu s'avère souvent plus coûteux que nécessaire, car les états intermédiaires ne nous intéressent pas.

		\paragraph{}
		Le procédé de plus courant est donc d'utiliser une méthode introduisant un pseudo-temps (Pseudo-transient continuation method).
		On cherche à résoudre l'équation (\ref{eq:f=0}), et on dispose d'un itéré initial $\textup{w}_0$.
		On va chercher la solution comme étant la limite en $t = +\infty$ de
		\begin{equation}\label{eq:edp}
			\left\{\begin{aligned}
				\frac{\partial\textup{w}}{\partial t}\left(t, x\right) = \operatorname{f}\left(\textup{w}\right) \\
				\textup{w}\left(t_0, x\right) = \textup{w}_0\left(x\right)
			\end{aligned}\right.,\qquad \forall x\in\mathcal{D}, \forall t\in\left[t_0, +\infty\right[\ .
		\end{equation}
		La différence avec la résolution d'un problème instationnaire et que les états intermédiaires ne nous intéressent pas, et donc on peut se permettre de calculer des états non physiques, ou des transitions différentes du vrai système physique, du moment que l'état convergé est correct.


\section{Discrétisation numérique}

	\paragraph{}
	Lorsque l'on résout une équation aux dérivées partielles avec une méthode numérique, il est nécessaire de réaliser une discrétisation du problème, ne serait-ce que pour pouvoir représenter l'état du système dans un domaine borné, représentable dans la mémoire d'un ordinateur.
	En pratique, on réalise deux niveaux de discrétisation : temporel et spatial.


	\subsection{Discrétisation spatiale}

		\paragraph{}
		Ma thèse se concentre sur l'intégration temporelle de l'équation (\ref{eq:edp}), mais il est toutefois bon de rappeler ce qu'est la discrétisation spatiale.
		Puisque la résolution des équations est numérique, il faut pouvoir représenter les données dans la mémoire d'un ordinateur.
		La discrétisation spatiale consiste ainsi à diviser le domaine d'étude $\mathcal{D}$ en un ensemble de cellules qui forment un maillage.
		Les grandeurs physiques étudiées comme la vitesse du fluide, sa température, sa densité, ..., sont alors représentées dans chaque cellule, par leur valeur moyenne, ou leur valeur sur chaque interface, ou de manière plus complexe en fonction du choix de discrétisation pris par l'utilisateur.
		L'état physique dans l'ensemble du domaine est représenté non plus par une fonction de l'espace $\textup{w} : \mathcal{D} \to \mathbb{R}^n$ mais par un vecteur d'états $W \in \mathbb{R}^{N\times n}$, en notant $n$ le nombre de degrés de liberté et $N$ le nombre de cellules du maillage.
		Ce vecteur est en fait constitué de l'ensemble des vecteurs d'états en l'ensemble des points du maillage.

		\paragraph{}
		La discrétisation spatiale donne également un moyen de calculer les dérivées spatiales de l'état en fonction de l'état.
		Pour mieux décrire ceci, prenons le cas utilisé dans ma thèse : la formulation volumes finis \cite{EymardGallouetHerbin2000}.
		Cette formulation ne s'applique que lorsque l'équation (\ref{eq:edp_0}) est une loi de conservation, c'est à dire que le second membre s'exprime comme la divergence d'un flux :
		\[\operatorname{f}\left(\textup{w}\right) = \nabla\operatorname{g}\left(\textup{w}\right)\ .\]
		La méthode des volumes finis consiste à intégrer l'équation (\ref{eq:edp}) pour chaque cellule du maillage.
		En notant $\mathcal{V}_i$ le volume et $\mathcal{S}_i$ la surface de la cellule $i$:
		\begin{align*}
			&&\int_{\mathcal{V}_i}\frac{\partial\textup{w}}{\partial t}\mathrm{d}v &= \int_{\mathcal{V}_i}\nabla\operatorname{g}\left(\textup{w}\right)\mathrm{d}v \\
			\Rightarrow&&\frac{\mathrm{d}}{\mathrm{d}t}\int_{\mathcal{V}_i}\textup{w}\mathrm{d}v &= \oint_{\mathcal{S}_i}\operatorname{g}\left(\textup{w}\right)\cdot\mathrm{d}s\quad\textrm{d'après le théorème de Stokes.}
		\end{align*}
		Si on note $\bar{\textup{w}}_i = \frac{1}{\mathcal{V}_i}\int_{\mathcal{V}_i}\textup{w}\mathrm{d}v$ la moyenne de l'état dans la cellule $i$, alors :
		\[\frac{\mathrm{d}\bar{\textup{w}}_i}{\mathrm{d}t} = \frac{1}{\mathcal{V}_i}\oint_{\mathcal{S}_i}\operatorname{g}\left(\textup{w}\right)\cdot\mathrm{d}s\ .\]

		\paragraph{}
		La formulation du schéma va ensuite indiquer la manière de calculer l'intégrale du flux sur la surface des cellules.
		L'important est que l'équation aux dérivées partielles est maintenant une équation différentielle ordinaire \cite{TrefethenBirkissonDriscoll2017}.
		Si on regroupe dans un seul grand vecteur les vecteurs d'états de chaque cellule, on peut de même regrouper les seconds membres ensemble pour n'avoir qu'une équation.
		Ainsi l'équation aux dérivées partielles (\ref{eq:edp}) peut se réécrire comme une équation différentielle ordinaire :
		\begin{equation}\label{eq:edo}
			\left\{\begin{aligned}
				&\frac{\mathrm{d}W}{\mathrm{d}t}\left(t\right) = F\left(W\right) \\
				&W\left(t_0\right) = W_0\in\mathbb{R}^{N\times n}
			\end{aligned}\right.\ .
		\end{equation}
		Il est important de préciser que cette nouvelle équation (\ref{eq:edo}) est purement numérique, contrairement à l'équation (\ref{eq:edp_0}) qui découlait de la physique.
		Puisque ma thèse ne s'intéresse pas à la discrétisation spatiale, nous nous placerons dans ce cas général et chercherons à résoudre l'équation (\ref{eq:edo}).

	\subsection{Discrétisation temporelle}

		\paragraph{}
		La discrétisation temporelle permet de représenter le temps continu par une succession de temps discrets.
		On va donc représenter et calculer la solution non pas sur l'ensemble du temps mais en ces instants discrets.
		Concrètement, à l'itération $n+1$ on transforme l'équation différentielle en relation de récurrence, entre l'état suivant à calculer $W_{n+1}$, l'état actuel $W_n$ et d'éventuels états précédents $W_{i,i<n}$.

		\paragraph{}
		Si on peut exprimer l'état suivant $W_{n+1}$ directement à partir de l'état courant (et éventuellement des précédents), la méthode d'intégration temporelle est dite explicite.
		L'intérêt d'une telle méthode est qu'elle permet de calculer l'état suivant très rapidement, souvent par la simple évaluation d'une fonction.
		Cependant, ces méthodes ont tendance à devenir instable dès que le nombre de Courant devient un peu grand (typiquement 1) et imposent donc l'utilisation de très faibles pas de temps.
		Puisque l'on cherche la solution au bout d'un temps long pour le problème stationnaire, l'utilisation de ces méthodes ne s'avère pas rentable : il vaut mieux payer plus cher l'itération mais converger plus rapidement, et c'est pour cela qu'on utilise les méthodes implicites.
		Elles se caractérisent par le fait que l'état suivant ne s'exprime pas explicitement à partir des états connus, mais qu'il est solution d'une équation, par exemple le zéro d'une fonction.


\chapter{Intégration temporelle}

\paragraph{}
Nous avons vu dans l'introduction que lorsqu'on résout un problème de simulation numérique de la dynamique des fluides, on se ramène à résoudre une équation différentielle ordinaire de la forme (\ref{eq:edo}).
Il existe différents algorithmes permettant de résoudre une telle équation, et la plupart de ces algorithmes se rangent dans deux catégories : les méthodes explicites, et les méthodes implicites.

\section{Analyse des méthodes}

  \paragraph{}
  Il est nécessaire de définir quelques notions pour pouvoir analyser les méthodes qui vont suivre.

  \section{Consistance et ordre}

    \paragraph{}
    Une méthode de résolution d'équations différentielles doit respecter certaines propriétés pour être "correcte".
    Notamment elle se doit d'être consistante.
    Pour définir cette notion, on se place dans le cadre de l'équation (\ref{eq:edo}).
    Après un pas, la méthode numérique donne une valeur $W_1$ que l'on souhaite proche de la valeur exacte $W\left(t_0 + \Delta\right)$.
    La méthode est dite consistante si :
    \[\lim_{\Delta t \rightarrow 0} \frac{W_1 - W\left(t_0 + \Delta t\right)}{\Delta t} = 0\]

    \paragraph{}
    De plus, on dira que la méthode est d'ordre $p$ si l'erreur locale est en $\Delta t^{p+1}$ :
    \[W_1 - W\left(t_0 + \Delta t\right) = O\left(\Delta t^{p+1}\right)\]


  \section{Stabilité}

    \paragraph{}
    Un critère important dans le choix d'une méthode d'intégration est la stabilité.
    En fonction de notre cas d'application, on exigera certains niveaux de stabilité pour éviter une divergence d'origine numérique du calcul.
    Nous cherchons dans cette section à analyser la stabilité d'une méthode d'intégration de l'équation (\ref{eq:edo}).
    Pour simplifier, on considère ici que le pas de temps $\Delta t = t_{n+1} - t_n$ ne dépend pas du numéro de l'itération $n$.
    La taille du problème est notée $N$.

    \paragraph{}
    Pour étudier la stabilité d'une méthode, on utilise en général l'équation différentielle ordinaire avec un second membre linéaire \cite{HairerWanner1996}.
    En effet, si on dispose d'une solution $\tilde{W}$ de (\ref{eq:edo}), on peut linéariser $F$ en $\tilde{W}$ :
    \[\frac{\mathrm{d}W}{\mathrm{d}t} = F\left(\tilde{W}\right) + \frac{\partial F}{\partial W}\left(\tilde{W}\right)\cdot\left(W - \tilde{W}\right)\]
    Si on note ensuite $y = W - \tilde{W}$ et $A = \frac{\partial F}{\partial W}\left(\tilde{W}\right)$, on obtient :
    \begin{equation}\label{eq:stab}
      \frac{\mathrm{d}y}{\mathrm{d}t} = Ay
    \end{equation}
    C'est donc avec cette équation, également appelée équation de Dahlquist, que l'on étudie la stabilité de la méthode.
    Nous l'étudions dans $\mathbb{C}$.

    \paragraph{}
    Pour une méthode d'intégration donnée, on note :
    \begin{equation}\label{eq:stab_req}
      y_{n+1} = G\left(\Delta tA\right)y_n
    \end{equation}
    la relation qui donne l'état suivant en fonction de l'état actuel.
    Pour la plupart des méthodes d'intégration temporelle, et en particulier pour celles présentées ici, on peut se ramener à cette écriture avec $G$ une fonction analytique.
    On peut choisir une base de vecteurs propres de $A$ $v_1, \dots, v_N$ associés aux valeurs propres $\alpha_1, \dots, \alpha_N$.
    Si on décompose l'itéré initial sur cette base, c'est à dire $y_0 = \sum_{i=1}^N\lambda_iv_i$, on voit que puisque $G$ est une fonction analytique :
    \[y_n = \sum_{i=1}^N\lambda_iG\left(\Delta t\alpha_i\right)^nv_i\]

    \paragraph{}
    Il découle de l'équation précédente que si $G\left(\Delta t\alpha_i\right)^n$, alors $y_n$ tend vers 0.
    C'est ainsi qu'on définit le domaine de stabilité d'une méthode d'intégration temporelle :
    \[\left\{\,z\in\mathbb{C}\;\mid\;\left|G\left(z\right)\right| < 1\,\right\}\]
    Si toutes les valeurs propres de $A$ sont à partie réelle négatives, alors la solution de l'équation (\ref{eq:stab}) converge vers 0.
    On comprend alors tout de suite que pour que l'algorithme soit stable, il faut que les valeurs propres de $A$ soient toutes dans le domaine de stabilité de la méthode.

    \paragraph{}
    On remarque que l'argument de $G$ n'est pas $A$ mais $\Delta tA$.
    Si une valeur propre $\Delta t\alpha_i$ de $\Delta tA$ n'est pas dans le domaine de stabilité de la méthode, la direction propre associée sera amplifiée et une instabilité d'origine numérique entraînera la divergence du calcul.
    On va donc jouer sur la valeur de $\Delta t$ : en la prenant assez faible, on arrivera à faire rentrer toutes les valeurs propres dans le domaine de stabilité, et donc à garantir la stabilité du calcul.
    Cependant, comme nous le verrons par la suite, cela impose l'utilisation de pas de temps relativement faible, ce qui s'avère contraignant pour nos application.

    \paragraph{}
    La propriété clef découlant de cette étude de stabilité est la "A-stabilité" \cite{Dahlquist1963}.
    Une méthode d'intégration temporelle est dite A-stable si son domaine de stabilité inclut le demi-plan des complexe à partie réelle négative.
    Autrement dit, une méthode est A-stable si elle converge bien vers 0 lorsqu'elle le devrait, et ne fait donc pas diverger l'erreur numérique.


\section{Méthodes explicites}

  \paragraph{}
  Les méthodes explicites cherchent l'état suivant $W_{n+1}$ en fonction des états précédents :
  \[W_{n+1} = F_e\left(W_n, W_{n-1}, \dots, W_0\right)\]
  Ces méthodes sont généralement utilisées dans les simulations numériques instationnaires de la dynamique des fluides.
  L'avantage de ces méthodes est d'une part leur facilité d'implémentation, et d'autre part leur faible coût numérique.
  En effet, puisque l'état recherché est accessible à partir des états précédents, la seule écriture d'une fonction est nécessaire au développement de la méthode, et le coût numérique à chaque itération se résume à l'évaluation de cette fonction.

  \subsection{Méthode d'Euler}

    \paragraph{}
    La méthode d'Euler, ou Euler explicite, est la méthode la plus simple qui soit.
    Elle consiste à intégrer (\ref{eq:edo}) entre $t_n$ et $t_{n+1} = t_n + \Delta t$ en supposant le second membre constant égal à $F\left(W_n\right)$, ou de manière équivalente à remplacer la dérivée dans (\ref{eq:edo}) par le taux d'accroissement : $\frac{\mathrm{d}W}{\mathrm{d}t}\approx\frac{W_{n+1}-W_n}{\Delta t}$.
    Ainsi, cette méthode donne la relation de récurrence explicite :
    \[W_{n+1} = W_n + \Delta t F\left(W_n\right)\]

    \paragraph{}
    Lorsqu'on réalise l'analyse de stabilité de cette méthode, on voit que la fonction $G$ de l'équation (\ref{eq:stab_req}) est $z\mapsto 1 + z$.
    Dans le plan complexe, son domaine de stabilité est donc le disque unité ouvert centré en -1.
    En pratique, ce domaine de stabilité s'avère insatisfaisant car il prohibe l'utilisation de grands pas de temps.


  \subsection{Méthodes de Runge-Kutta}

    \paragraph{}
    Les méthodes de Runge-Kutta forment une classe de méthodes qu'on appellera "multi-étapes", pour établir un contraste avec les méthodes "multi-pas" présentées plus bas.
    Plutôt que de faire un unique pas en avant comme le fait la méthode d'Euler explicite, on va réaliser un ensemble de pas intermédiaires, le pas final étant une combinaison de ces pas intermédiaires.

    \paragraph{}
    Le principe général est le suivant.
    En supposant qu'en $t_n$ on dispose d'une valeur de la solution $W_n$, on introduit les étapes intermédiaires $t_{n,i} = t_n + c_i\Delta t$ pour $1\leq i\leq k$, avec $k$ fixé.
    On peut intégrer entre $t_n$ et $t_{n,i}$ l'équation (\ref{eq:edo}) de manière exacte :
    \[W\left(t_{n,i}\right) = W\left(t_n\right) + \Delta t \int_0^{c_i}F\left(W\left(t_n + s\Delta t\right)\right)\mathrm{d}s\]
    L'intégrale est ensuite approchée par une quadrature sur les points $j<i$ :
    \[\int_0^{c_i}F\left(W\left(t_n + s\Delta t\right)\right)\mathrm{d}s \approx \sum_{j = 1}^{i-1}a_{ij}F\left(W\left(t_{n,j}\right)\right)\]
    Les étapes déjà calculées permettent donc de calculer la quadrature afin d'obtenir l'étape suivante.
    Une fois qu'on a obtenu toutes les étapes intermédiaires, on intègre (\ref{eq:edo}) entre $t_n$ et $t_{n+1}$ :
    \[W_{n+1} = W_n + \Delta t \int_0^1F\left(W\left(t_n + s\Delta t\right)\right)\mathrm{d}s\]
    que l'on approche de nouveau par la quadrature :
    \[\int_0^1F\left(W\left(t_n + s\Delta t\right)\right)\mathrm{d}s \approx \sum_{i = 1}^kb_iF\left(W\left(t_{n,i}\right)\right)\]

    \paragraph{}
    Une méthode de Runge-Kutta est donc caractérisée par sa taille $k$ et par les valeurs $a_{ij, 1\leq j<i\leq k}$, $b_{i, 1\leq i\leq k}$ et $c_{i, 1\leq i\leq k}$.
    Ces valeurs de quadrature sont souvent représentées dans le tableau de Butcher :
    \[\begin{array}{c|ccccc}
      0\\
      c_2    & a_{21} \\
      c_3    & a_{31} & a_{32} \\
      \vdots & \vdots &        & \ddots\\
      c_k    & a_{k1} & a_{k2} & \hdots & a_{k,k-1}\\
      \hline & b_1    & b_2    & \hdots & b_{k-1} & b_k
    \end{array}\]

    On rappelle ici les tableau de Butcher des méthodes les plus connues :
    \begin{table*}[h]\begin{tabular}{P{.15\textwidth}P{.3\textwidth}P{.4\textwidth}}
      \begin{tabular}{c|c}
        0 \\ \hline & 1
      \end{tabular} &
      \begin{tabular}{c|cc}
        0 \\ 1/2 & 1/2 \\ \hline & 0 & 1
      \end{tabular} &
      \begin{tabular}{c|cccc}
        0 \\ 1/2 & 1/2 \\ 1/2 & 0 & 1/2 \\ 1 & 0 & 0 & 1 \\ \hline & 1/6 & 1/3 & 1/3 & 1/6
      \end{tabular} \\
      RK1 & RK2 & RK4 \\
    \end{tabular}\end{table*}

    \paragraph{}
    La méthode RK1 est équivalente à la méthode Euler explicite présentée précédemment.
    Cette méthode RK2 est également appelée méthode du point milieu.
    Cette méthode RK4 est d'ordre 4, et est l'une des plus utilisées pour l'intégration explicite des équations différentielles.

    \paragraph{}
    On peut montrer que l'ordre de la méthode $p$ est tel que $p \leq k$.
    Jusqu'à $k = 4$ on a $k = p$.
    Pour les ordres supérieurs, obtenir une borne entre l'ordre et la taille de la méthode est encore un problème ouvert.
    Pour une méthode de Runge-Kutta d'ordre $p$, la fonction $G$ de l'équation (\ref{eq:stab_req}) est telle que \cite{HairerWanner1996}:
    $$G\left(z\right) = 1 + z + \frac{z^2}{2} + \dots + \frac{z^p}{p!} + O\left(z^{p+1}\right)$$
    Lorsque la méthode est telle que l'ordre $p$ est égal à la taille $k$, le terme en $O\left(z^{p+1}\right)$ est nul.

  \subsection{Méthodes d'Adams-Bashforth}

    \paragraph{}
    Les méthodes d'Adams-Bashforth sont des méthodes de type "multi-pas" (multi-step), c'est à dire qu'elles utilisent plusieurs états précédents pour déterminer l'état suivant.
    Pour une méthode d'ordre $k$, on utilise donc les états $W_n, \dots, W_{n-k+1}$ pour déterminer $W_{n+1}$.
    On peut en effet vérifier que pour ces méthodes, l'indice $k$ désignant la méthode est égal à l'ordre de la méthode \PS{Citer "Solving ordinary differential equations I: Nonstiff problems" si j'arrive à le trouver}.

    \paragraph{}
    L'idée est d'interpoler le second membre de (\ref{eq:edo}) en ces $k$ points calculés précédemment.
    Il existe en effet un unique polynôme $F_k$ tel que pour $0 \leq i < k$ on ait $F_k\left(t_{n-i}\right) = F\left(W_{n-i}\right)$.
    On fait ensuite l'hypothèse $F_k\left(t\right) \approx F\left(W\left(t\right)\right)$.
    On peut intégrer l'équation (\ref{eq:edo}) :
    \[W_{n+1} = W_n + \int_{t_n}^{t_{n+1}}F_k\left(t\right)\mathrm{d}t\]

    \paragraph{}
    Contrairement aux méthodes de Runge-Kutta, puisqu'on réutilise les pas précédents, une unique évaluation du second membre de (\ref{eq:edo}) est nécessaire à chaque itération.
    Le coût de calcul est donc très faible pour de telles méthodes.
    Cependant, les propriétés de stabilité de ces méthodes sont moindres, ce qui explique le fait qu'elles ne sont pas souvent utilisées en pratique dans des codes de calcul pour la dynamique des fluides.
    De manière générale, il n'existe pas de méthodes multi-pas explicites qui sont A-stables.


\section{Méthodes implicites}


\chapter{Résolution du système linéaire}

\paragraph{}
On s'intéresse dans ce chapitre à la résolution numérique d'un système linéaire $Ax = b$, de taille $N$.
On cherche donc $x\in\mathbb{K}^N$ avec le second membre $b\in\mathbb{K}^N$ et la matrice $A\in\matrixsymb{N}{K}$ connus.
Dans le cadre de l'énergétique et de la multi-physique, les équations sont réelles mais pour plus de généralité dans la suite on prend le corps $\mathbb{K} = \mathbb{R}\textrm{ ou }\mathbb{C}$.


\section{Système linéaire creux}

	\paragraph{}
	Comme cela a été mentionné, la taille des matrices rencontrées dans un problème typique de la simulation numérique de la dynamique des fluides est très grande.
	Les matrices ont également une autre propriété : la creusité.
	Puisque les matrices des systèmes linéaires à résoudre sont liées à une matrice jacobienne (voir équation (\ref{eq:linear})), un coefficient de la matrice symbolise une interaction entre deux points du maillage.
	On comprend alors que si le graphe de ces interactions dépend du schéma de discrétisation et d'intégration spatiale, deux points éloignés dans le maillage ne seront normalement pas reliés.
	Ainsi, les matrices présentent cette caractéristique notable qu'est la creusité et que l'on peut observer sur la figure \ref{fig:sparse}.
	On y voit en effet que la majorité des coefficients du système linéaire \GP{est} 0.
	En pratique, les matrices rencontrées sont tellement grandes qu'il n'est pas possible de les stocker et de les utiliser sous leur forme dense.
	La propriété de creusité permettra d'utiliser des représentation astucieuses pour stocker les matrices en mémoire, telle que le format CSR (Compressed Sparse Row) \cite{Saad2003}.
	Ces formats permettent à la fois d'économiser de l'espace en mémoire lors du stockage de la matrice, mais également de réaliser des opérations telles que le produit matrice vecteur plus efficacement qu'avec une matrice dense de même taille.

	\begin{figure}
		\centering
		\begin{subfigure}[t]{0.3\textwidth}
			\centering
			\includegraphics[width=\textwidth]{images/GT01R.png}
			\caption{GT01R : écoulement 2D non visqueux à travers un étage d'aubes de turbomachine}
			\label{fig:sparse.GT01R}
		\end{subfigure}
		\hfill
		\begin{subfigure}[t]{0.3\textwidth}
			\centering
			\includegraphics[width=\textwidth]{images/HV15R.png}
			\caption{HV15R : écoulement RANS 3D dans la soufflante d'un moteur}
			\label{fig:sparse.HV15R}
		\end{subfigure}
		\hfill
		\begin{subfigure}[t]{0.3\textwidth}
			\centering
			\includegraphics[width=\textwidth]{images/RM07R.png}
			\caption{RM07R : écoulement 3D visqueux dans le compresseur d'un turbopropulseur}
			\label{fig:sparse.RM07R}
		\end{subfigure}
		\caption{Représentation de matrices issues de problèmes de CFD \cite{PacullAubertBuisson2011}, utilisant une méthode spatiale Volumes Finis, les points de couleurs correspondant aux valeurs non nulles}
		\label{fig:sparse}
	\end{figure}


\section{Type de méthode}

	\paragraph{}
	Il existe un grand nombre de méthodes pour inverser un problème linéaire.
	Cependant, en simulation numérique de la dynamique des fluides, la taille des maillage est souvent très importante (\PS{PAR EXEMPLE ?}), ce qui engendre des systèmes de très grande taille.
	Certaines méthodes de résolution ne sont alors plus compatibles, comme en particulier les méthodes directes.
	Cette famille de méthodes, dont fait partie par exemple la méthode du pivot de Gauss, calcule exactement la solution du système linéaire, mais nécessite un temps de calcul très important et un espace mémoire déraisonnable pour les tailles de nos problèmes.
	À l'inverse, les méthodes itératives produisent une suite de vecteurs qui convergent vers la solution du système linéaire.
	L'erreur sur la résolution du système décroît à mesure que l'algorithme itère et cela permet d'atteindre la précision souhaitée, non nulle mais suffisamment petite, pour un temps de calcul et une consommation mémoire maîtrisé.
	C'est l'idée que l'on retrouve sur la figure \ref{fig:direct-iterative} : la méthode directe ne fait aucun progrès avant $O\left(N^3\right)$ opérations, alors que la norme de l'erreur décroît avec le numéro d'itération pour la méthode itérative.

	\begin{figure}
		\centering
		\includegraphics[width=.7\textwidth]{images/direct-iterative.png}
		\caption{Convergence des méthodes directes et itératives, issu de \cite{TrefethenBau1997}}
		\label{fig:direct-iterative}
	\end{figure}

	\paragraph{}
	Les méthodes itératives sont donc les plus adaptées pour inverser des systèmes linéaires creux de grande taille.
	Cependant, il peut arriver qu'on utilise une méthode directe : on peut envisager par exemple la résolution d'un système linéaire avec une succession d'étages d'un algorithme multigrille, pour arriver à un système grossier de petite taille (ou du moins de taille raisonnable), qu'on résoudrait avec une méthode directe comme la décomposition LU.
	Ces méthodes sont bien connues \cite{TrefethenBau1997}, et ne seront pas détaillées ici.


\section{Méthodes itératives}

	\paragraph{}
	Les méthodes itératives sont encore aujourd'hui très étudiées dans la communauté des mathématiques appliquées \cite{OlshanskiiTyrtyshnikov2014, Saad2003, TrefethenBau1997}.
	Une première classe de méthode intervient : les méthodes itératives classiques.
	Si ces méthodes sont plutôt anciennes et plutôt écartées pour leurs mauvaises performances, elles peuvent s'utiliser en se combinant avec des méthodes plus efficaces, et c'est pour cela qu'il est important de comprendre leur fonctionnement.
	Ce sont des méthodes de relaxation, parmi lesquelles on compte les méthode de Jacobi, Gauss-Seidel ou encore SOR (Successive Over Relaxation).
	Leur fonctionnement réside dans une décomposition $A = M - N$ avec $M$ une matrice relativement facile à inverser, où le choix de la décomposition est propre à la méthode.
	Pour la méthode de Jacobi on prendra pour $M$ la diagonale de $A$.
	Il suffit ensuite, à partir d'un itéré initial $x_0$ d'appliquer la formule de récurrence $x_{k+1} = M^{-1}\left(Nx_k + b\right)$.
	Cependant, comme dit précédemment, ces méthodes n'offrent pas une convergence satisfaisante pour la communauté de CFD.
	Ainsi, il faut se tourner vers un autre type de méthodes itératives pour résoudre le système linéaire.


\section{Méthodes de Krylov}

	\paragraph{}
	Les méthodes de Krylov s'imposent comme étant la norme dans le domaine de la dynamique des fluides numérique\footnote{\GP{Qui l'a dit? Ou as-tu trouve la reference?}}.
	Le principe d'une méthode de Krylov est de projeter le système linéaire sur un espace restreint appelé espace de Krylov, et d'y résoudre le système maintenant qu'il est plus petit.
	L'astuce va être dans le fait que ces espaces de Krylov sont emboîtés, et qu'à chaque nouvelle itération on utilise l'information accumulée lors des précédentes.

	\paragraph{}
	Dans ce qui suit, on notera à l'itération $n$ le résidu $r_n = b - Ax_n$, où $x_n$ est l'itéré produit par l'algorithme.
	Dans la pratique, on se restreint à un nombre d'itérations $n\ll N$ pour des raisons évoquées précédemment, mais la suite reste vraie pour $n\le N$.
	On supposera également que l'équation (\ref{eq:linear}) admet une solution.
	Enfin, on notera que résoudre $Ax = b$ avec un itéré initial $x_0$ est équivalent ici à résoudre $A\tilde{x} = b - Ax_0$ avec un itéré initial nul, puis à prendre $x_0 + \tilde{x}$ comme vecteur solution.

	\paragraph{}
	Pour résoudre $Ax = b$ en partant d'un itéré initial $x_0$ et donc d'un résidu $r_0 = b - Ax_0$, on définit l'espace de Krylov à l'itération $n$:
	\begin{equation}\label{eqn:krylov}
		\krylov[A, r_0]{n} = \operatorname{Vect}\left(r_0, Ar_0, \dots, A^{n-1}r_0\right)\ .
	\end{equation}
	On constate alors que par définition $\krylov{1}\subset\krylov{2}\subset\dots\subset\krylov{n-1}\subset\krylov{n}$.
	On cherche ensuite l'itéré sur $\krylov{n}$ qui garantit une condition de Petrov-Galerkin \cite{SimonciniSzyld2007} :
	\begin{equation}\label{eqn:x_n}
		x_n \in x_0 + \krylov[A, r_0]{n}\quad\textrm{tel que}\quad x_n \perp \mathcal{L}_n
	\end{equation}
	où $\mathcal{L}_n$ est un sous-espace vectoriel de dimension $n$.
	Pour $\mathcal{L}_n = \krylov{n}$ on parle de condition de Galerkin, et pour $\mathcal{L}_n = A\krylov{n}$ de minimisation du résidu.
	On voit donc que si $A$ est inversible, $\mathcal{L}_N = \mathbb{K}^N$ et alors l'algorithme termine en au plus $N$ itérations.
	Cette remarque n'a pas grand intérêt en pratique puisqu'on arrêtera l'algorithme bien avant.

	\paragraph{}
	Pour construire les différents espaces de Krylov, on utilise l'itération d'Arnoldi \cite{TrefethenBau1997}.
	La matrice $A$ est semblable à une matrice de Hessenberg $H$ :
	\begin{equation}\label{eq:hessenberg}
		A = VH\adj{V}\Leftrightarrow AV = VH\quad\textrm{avec}\quad H=
		\begin{pmatrix}
			h_{1,1}&h_{1,2}&h_{1,3}&\cdots &h_{1,N}\\
			h_{2,1}&h_{2,2}&h_{2,3}&\cdots &h_{2,N}\\
			0      &h_{3,2}&h_{3,3}&\cdots &h_{3,N}\\
			\vdots &\ddots &\ddots &\ddots &\vdots \\
			0      &\cdots &0      &h_{N,N-1}&h_{N,N}
		\end{pmatrix}\ .
	\end{equation}
	Dans (\ref{eq:hessenberg}), $V$ est une matrice orthonormale de taille $N$.
	On notera $v_i$ sa $i^{\textrm{ème}}$ colonne.
	On notera également $V_n\in\matrixsymb{N, n}{K}$ la matrice constituée des $n$ premières colonnes de $V$.
	On notera enfin $\widetilde{H}_n\in\matrixsymb{n+1, n}{K}$ le bloc supérieur gauche de $H$, qui est également de Hessenberg.

	\paragraph{}
	En considérant les $n \le N$ premières colonnes de (\ref{eq:hessenberg}), de par la structure de Hessenberg de $H$, on obtient la relation d'Arnoldi :
	\begin{equation}\label{eq:arnoldi}
		AV_n = V_{n+1}\widetilde{H}_n
	\end{equation}
	et en considérant la dernière colonne de l'équation (\ref{eq:arnoldi}) :
	\begin{equation}\label{eq:arnoldi_n+1}
		Av_n = h_{1,n}v_1 + \dots + h_{n,n}v_n + h_{n+1,n}v_{n+1}\ .
	\end{equation}

	\paragraph{}
	On peut maintenant décrire l'algorithme :
	Puisque $\krylov{1} = \operatorname{Vect}\left(r_0\right)$, on pose $v_1 = \frac{r_0}{\norm{r_0}}$.
	On itère ensuite sur $n$ : en connaissant $V_n$, la relation (\ref{eq:arnoldi_n+1}) permet de trouver le nouveau vecteur de base et la $n$\textsuperscript{ième} colonne de $H$, c'est à dire de trouver $V_{n+1}$ et $\widetilde{H}_n$.
	Les espaces vectoriels $\operatorname{Im}\left(V_n\right)$ sont alors par construction les espaces de Krylov $\krylov{n}$.

	\paragraph{}
	On remarque que la matrice $A$ n'intervient dans la relation (\ref{eq:arnoldi_n+1}) qu'à travers le produit $Av_n$.
	Il n'est donc pas nécessaire de former explicitement la matrice du système linéaire, mais seulement de pouvoir calculer son action sur un vecteur.
	Cette propriété des méthodes de Krylov sera utilisée par la suite.

	\paragraph{}
	Il existe donc beaucoup de méthodes de Krylov pour résoudre un système linéaire, la principale distinction entre elle se faisant par le choix du sous-espace $\mathcal{L}_n$.
	La condition de minimisation du résidu donne la méthode GMRES, qui est l'une des plus utilisée par les communautés de CFD et de mathématiques appliquées.
	Elle permet, après des itérations un peu plus coûteuses que ses consœurs telles que la méthode Bi-CG \cite{TrefethenBau1997}, de minimiser le résidu obtenu et donc de garantir la décroissance de l'erreur.
	De plus, la méthode GMRES est beaucoup étudiée et de nombreuses variantes et améliorations ont été développées au fil des années.
	Pour ces raisons, et parce que son implémentation est relativement simple, c'est sur cette méthode que j'ai décidé de m'orienter.


\section{GMRES}

	\paragraph{}
	La méthode GMRES consiste donc, à chaque itération $n$, à :
	\begin{itemize}
		\item calculer le nouvel espace de Krylov $\krylov{n+1}$ à partir de $\krylov{n}$, donc d'après (\ref{eq:arnoldi_n+1}) de calculer le produit $Av_n$, puis de l'orthonormaliser sur les colonnes de $V_n$ pour obtenir $\widetilde{H}_n$ et $V_{n+1}$
		\item trouver $x_n \in x_0 + \krylov{n}$ qui minimise la norme de $r_n = b - Ax_n$.
	\end{itemize}
	Puisque $x_n \in x_0 + \krylov{n}$, $x_n = x_0 + V_n y_n$ pour un certain $y_n\in\mathbb{K}^n$.
	On dispose de la relation d'Arnoldi (\ref{eq:arnoldi}), donc :
	\begin{align*}
		\norm{r_n} &= \norm{b - Ax_n} \\
		&= \norm{r_0 - AV_ny_n} \\
		&= \norm{r_0 - V_{n+1}\widetilde{H}_ny_n} \\
		&= \norm{\norm{r_0} e_1 - \widetilde{H}_ny_n}\qquad\textrm{avec $e_1$ le premier vecteur de la base canonique de $\mathbb{K}^{n+1}$.}
	\end{align*}

	\paragraph{}
	La condition de minimisation du résidu est remplie en résolvant un problème de minimisation de taille $n$.
	En pratique ce problème est résolu par décomposition QR de la matrice de Hessenberg.
	Pour ce faire, on suppose qu'on dispose d'une décomposition $\Omega_n\widetilde{H}_n = \begin{pmatrix}R_n \\ 0\end{pmatrix}$ avec $\Omega_n\in\orthogroup{n+1}{K}$ une matrice unitaire et $R_n\in\matrixsymb{n}{K}$ une matrice diagonale.
	La décomposition QR de $\widetilde{H}_n$ à en effet cette forme de par sa structure de Hessenberg.
	C'est le cas à l'itération 1 avec $\Omega_1 = \operatorname{Id}$ et $R_1 = 1$.
	On note ensuite :
	\[\begin{pmatrix}\Omega_n & 0 \\ 0 & 1\end{pmatrix}\widetilde{H}_{n+1} = \begin{pmatrix}R_n & r_{n+1} \\ 0 & \rho \\ 0 & \sigma\end{pmatrix} \quad\textrm{avec}\quad \begin{pmatrix}r_{n+1} \\ \rho\end{pmatrix} = \Omega_n\begin{pmatrix}h_{n,n+1} \\ \vdots \\ h_{n+1,n+1}\end{pmatrix} \quad\textrm{et}\quad \sigma = h_{n+2,n+1}\ .\]
	On pose ensuite la rotation de Givens :
	\[G_n = \left\{\begin{aligned}
		&\quad\operatorname{Id}_{n+2} &\textrm{si }\rho = \sigma = 0\\ \\
		\begin{pmatrix}\operatorname{Id}_n & 0 & 0 \\ 0 & c & s \\ 0 & -s & c\end{pmatrix}& \quad\textrm{avec}\quad c = \frac{\rho}{\sqrt{\rho^2 + \sigma^2}}, s = \frac{\sigma}{\sqrt{\rho^2 + \sigma^2}} &\quad\textrm{sinon.}
	\end{aligned}\right.\]
	On remarque enfin que :
	\[G_n\begin{pmatrix}\Omega_n & 0 \\ 0 & 1\end{pmatrix}\widetilde{H}_{n+1} = \begin{pmatrix}R_n & r_{n+1} \\ 0 & \sqrt{\rho^2 + \sigma^2} \\ 0 & 0\end{pmatrix}\]
	ce qui donne la nouvelle décomposition QR : $\Omega_{n+1}\widetilde{H}_{n+1} = \begin{pmatrix}R_{n+1} \\ 0\end{pmatrix}$.


	\paragraph{}
	Finalement, la décomposition QR de la matrice de Hessenberg se construit à mesure des itérations, de manière peu coûteuse.
	Cette décomposition est cependant très intéressante.
	Lorsqu'on cherche à minimiser le résidu, on à montré précédemment qu'on minimise :
	\begin{align*}
		\norm{r_n} &= \norm{\norm{r_0} e_1 - \widetilde{H}_ny_n} \\
		&= \norm{\norm{r_0} \Omega_ne_1 - \begin{pmatrix}R_ny_n \\ 0\end{pmatrix}}\ .
	\end{align*}
	La solution de ce problème de minimisation est connu.
	En notant $\Omega_ne_1 = \begin{pmatrix}\omega_1 \\ \omega_{1n}\end{pmatrix}$ la première colonne de $\Omega_n$, on a :
	\[y_n = \norm{r_0}R_n^{-1}\omega_1\quad\textrm{et}\quad\norm{r_n} = \norm{r_0}\left|\omega_{1n}\right|\ .\]
	L'intérêt de mettre à jour la décomposition de la matrice de Hessenberg est alors double :
	\begin{itemize}
		\item l'inversion du système réduit est explicite, par inversion d'une matrice triangulaire ce qui est aisé
		\item la norme du résidu peut être calculée à chaque itération sans inverser le système, ce qui permet un meilleur contrôle des itérations.
	\end{itemize}

	\paragraph{}
	Mon choix s'est porté sur la méthode GMRES pour plusieurs raisons.
	Tout d'abord, c'est une méthode de Krylov, et le choix de ce type de méthode a déjà été argumenté précédemment.
	Ensuite, cette méthode permet un contrôle du résidu, car la norme de celui ci décroît entre chaque itération.
	Ce n'est pas le cas pour toutes les méthodes de Krylov comme par exemple pour la méthode Bi-CG.
	GMRES est un algorithme relativement simple qui ne nécessite pas (ou peu) de réglages, que ce soit de la part du développeur ou bien de l'utilisateur.
	Cette méthode de résolution du système linéaire est encore aujourd'hui bien présente au sein de la communauté de mathématiques appliquées \cite{Vasseur2016}, de CFD \cite{FrancoCamierAndrejEtAl2020} et est utilisée dans bien d'autres domaines, comme par exemple l'électromagnétisme \cite{ErnstGander2012} ou la mécanique du solide \cite{Mercier2015}.


\pagebreak

\chapter{TODO}


	%
	% 	\paragraph{}
	% 	Plus récemment, les méthodes au centre de l'attention sont les méthodes itératives de Krylov
	% 	\cite{TrefethenBau1997, Saad2003, SimonciniSzyld2007, OlshanskiiTyrtyshnikov2014, Vasseur2016}.
	%
	% 	\paragraph{}
	% 	Ces méthodes vont construire une base de l'espace de Krylov, en l'enrichissant à chaque itération, puis résoudre le système linéaire projeté sur cet espace.
	% 	Pour l'itération $k$, on appelle résidu le vecteur $r_k = b - Ax_k$.
	% 	On considère dans ce qui suit que la matrice $A$ est de taille $N\times N$,
	% 	et que dans l'esprit des algorithmes : $n\ll N$, même si la suite reste vraie avec $n$ quelconque.
	%
	%
	% 	\paragraph{}
	% 	À chaque itération de l'algorithme, on va calculer le vecteur de base à ajouter à $\krylov{n}$ pour engendrer $\krylov{n+1}$.
	% 	On cherche ensuite l'itéré sur $\krylov{n}$ qui s'approche de la solution voulue :
	% 	\begin{equation}\label{eqn:x_n}
	% 		x_n \in x_0 + \krylov[A, r_0]{n}
	% 	\end{equation}
	%
	% \paragraph{}
	% La convergence des méthode itératives, classiques comme de Krylov, est très fortement liée au conditionnement de l'opérateur $A$.
	% La définition du conditionnement dépend de la norme choisie, et pour la norme $\norm[2]{\cdot}$ on a, pour une matrice $A$ inversible :
	% \begin{equation}\label{eqn:conditionnement}
	% 	\kappa\left(A\right) = \frac{\sigma_{\max}}{\sigma_{\min}} = \frac{\left|\lambda_{\max}\right|}{\left|\lambda_{\min}\right|}
	% \end{equation}
	% où $\sigma_{\min}$ et $\sigma_{\max}$ sont les valeurs singulières minimales et maximales de $A$,
	% $\lambda_{\min}$ et $\lambda_{\max}$ sont les valeurs propres de $A$ de plus petit et de plus grand module,
	% et où $\kappa\left(A\right)\geq1$ est le conditionnement de $A$.
	% Une matrice bien conditionnée, du point de vue des algorithmes itératifs, a un conditionnement $\kappa\sim1$.
	% Au contraire, une matrice mal conditionnée a des valeurs propres éloignées, et $\kappa\gg1$.

%
% FGMRES \cite{SimonciniSzyld2002, Pinel2010, Vasseur2016, CoulaudGiraudRametEtAl2013}.
%
% Pas possible d'utiliser le spectre pour matrice non-normale \cite{GreenbaumStrakos1994, GreenbaumPtakStrakos1996, Trefethen1999}.
%
% Préconditionnement polynomial \cite{DuboisGreenbaumRodrigue1979}.
%
% GMRES augmentation/deflation \cite{ChapmanSaad1997, RamosKehlNabben, Pinel2010, Vasseur2016, Morgan2002, CoulaudGiraudRametEtAl2013}.
%
% AMG \cite{Vasseur2016}.
%
% JFNK \cite{LiuZhangZhongEtAl2015, Turpault2003, KnollKeyes2004},
%
% AD \cite{Griewank2000} utilisé en CFD \cite{BilanceriBeuxElmahiEtAl2011, KenwayMaderHeEtAl2019}.
%
% Autre que implicite classique :
% méthodes IMEX (rk \cite{HuangPerssonZahr2019}),
% relaxation \cite{CouletteFranckHelluyEtAl2019},
% multiphysique \cite{WongKwokHorneEtAl2019},
% exponentiel \cite{BhattKhaliqWade2018} implicite \cite{NieZhangZhao2006}.
%
% CEDRE \cite{Selva1998}.
%
% \section{Programmes et bibliothèques}
% 	\subsection{PETSc}
% 		\cite{petsc-web-page, petsc-user-ref, petsc-efficient}, PETSc TS \cite{AbhyankarBrownConstantinescuEtAl2018}
% 	\subsection{Maquette}
% 	\subsection{CEDRE}
%



\pagebreak
\bibliography{bibliography.bib}
\bibliographystyle{ieeetr}

\end{document}
