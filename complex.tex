\chapter{Méthodes plus complexes}

\paragraph{}
Dans les chapitres précédents, nous avons présentés des outils extraits de la littérature permettant de résoudre des équations différentielles.
Nous avons expliqué comment, en partant d'une équation aux dérivées partielles issue du modèle physique, nous arrivions après une étape de discrétisation numérique à une équation différentielle ordinaire.
Des méthodes d'intégration temporelle ont été présentée afin de résoudre cette équation, et nous avons mis en avant une classe d'intégrateurs : les méthode implicites.
Ces méthodes permettent de calculer l'état du système en résolvant des équations non linéaires, qui sont elles même résolue avec la solution de systèmes linéaires.
Enfin, nous avons identifié une méthode de Krylov, GMRES, et certaines de ses améliorations, pour pouvoir résoudre un système linéaire.

\paragraph{}
Ce cheminement permet l'intégration temporelle de phénomènes physiques, mais à nouveau ce n'est pas le seul possible.
On peut envisager d'utiliser des méthodes et des techniques moins classiques pour arriver à nos fins.


\section{JFNK}

  \paragraph{}
  Dans le cadre de la simulation numérique de grande dimension, une formulation apparaît comme très séduisante.
  Il s'agit des méthodes Jacobian Free Newton Krylov (JFNK), utilisées pour résoudre le système linéaire.
  Ces méthodes sont dites "sans matrices", car elles ne nécessitent pas la formation explicite de la matrice du système linéaire.
  Cela n'est pas anodin, car puisqu'on travaille sur des maillages de très grande taille, la matrice est donc de très grande dimension.
  La calculer a alors un certain coût algorithmique, et la stocker a un grand coût en mémoire.
